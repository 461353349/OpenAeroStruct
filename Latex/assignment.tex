\documentclass{article}
\usepackage{fullpage}
\usepackage{hyperref}

\title{MDO assignment; v. 1.0}
%\subtitle{Aerostructural optimization}
\author{John T. Hwang, Justin S. Gray, John P. Jasa, and Joaquim R. R. A. Martins}

\newcommand\be{\begin{enumerate}}
\newcommand\ee{\end{enumerate}}
\newcommand\code[1]{\texttt{#1}}

\begin{document}
	
	\maketitle

	\be

		\item \textbf{Structural optimization.} For this problem, we will use 
		\code{prob1.py} as our starting point.
		\be
			\item This script performs structural analysis and optimization of 
			a tubular beam clamped in the middle. 
			Run the optimization, first with uniform loading
			and then again with tip loads applied.
			What optimized thickness distributions do you see?\\
			Commands: 
			\be
				\item run the optimization: \code{python prob1.py}
				\item view the results: \code{python plot\_all.py s}
			\ee

			\item Run the optimization with tip loads applied for 
			a range of different mesh sizes (num\_y).
			Plot the computation time vs num\_y.
			\item The script produces an html file, \code{prob1.html},
			that can be useful for studying the problem structure. You can 
			open this file in any web-browser. 
			What is the physical interpretation of this problem?
			That is, what are we minimizing and subject to what constraint?
		\ee
		
		\item \textbf{Multidisciplinary analysis.} We now want to couple
		aerodynamics and structures together.
		\be
			\item Open \code{aerostruct.html} to use a guide. 
			Assemble the aerostructural analysis group following the layout presented 
			there. For this problem start with \code{prob2a.py}. 
			\item Try NLGS and Newton, and Hybrid NLGS/Newton for 
			mesh sizes (2,3), (4,5) and compare run times. Then try to run 
			the problem with the newton solver in the \code{root} group instead of the \code{coupled} group. 
			For this problem work with \code{prob2b.py}. Why do we put the nonlinear solver on the `coupled` group, 
			instead of the `root` group?
			\item (Bonus) Try LNGS, Krylov, Krylov-PC-GS, and direct linear solvers 
			with the Newton nonlinear solver. 
			Which ones can successfully converge the linear problem? 
			Which one gives the fastest convergence for the Newton solver?
			Why shouldn't we use the \code{DirectSolver} with high-fidelity problems? 
		\ee
		
		\item \textbf{Multidisciplinary optimization.} Now that you've worked 
		with the aerostructural analysis, you're ready to try aerostructural 
		optimization. 
		\be
			\item Compute the derivatives of the multidisciplinary system
			using finite differences by running \code{prob3ab.py}. Take note of the 
			run times and derivatives values output by the run script. 
			\item Now compute the same derivatives using the adjoint method
			and compare the timings for different mesh sizes. Use \code{prob3ab.py} again,
			but comment out the settings at the bottom of the file to switch on 
			analytic derivatives. 
			\item We will now perform aerostructural optimization.
			Edit \code{prob3c.py} by adding the following design variables:
			\begin{itemize}
				\item `twist', lower = -10, upper = 10, scaler = 1000
				\item `alpha', lower = -10, upper = 10, scaler = 1000
				\item `t', lower = 0.003, upper = 0.025, scaler = 1000
			\end{itemize}
			and the follow objective and constraints:
			\begin{itemize}
				\item `fuelburn'
				\item `failure', upper = 0
				\item `eq\_con', equals = 0
			\end{itemize}
			This optimization will take some time, but you can monitor the progess while it runs. 
			Without stopping the optimization, open a second command window and type the command: 

			\code{python Optview.py aerostruct.db}. 

			You can change the settings to adjust what variables you're plotting and you can check the 
			\code{Automatically refresh} option to have Optview update the plots
			as new iterations are saved. 

			You can also open a 3D visualization of your wing by typing the command: 

			\code{python plot\_all.py as}


		\ee
		
	\ee

\end{document}