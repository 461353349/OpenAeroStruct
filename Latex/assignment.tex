\documentclass{article}
\usepackage{fullpage}

\title{MDO assignment; v. 1.0}
%\subtitle{Aerostructural optimization}
\author{John T. Hwang, Justin S. Gray, and John Jasa}

\newcommand\be{\begin{enumerate}}
\newcommand\ee{\end{enumerate}}
\newcommand\code[1]{\texttt{#1}}

\begin{document}
	
	\maketitle

	\be

		\item \textbf{Structural optimization.} For this problem, we will use 
		\code{prob1.py} as our starting point.
		\be
			\item This script performs structural analysis and optimization of 
			a tubular beam clamped in the middle. 
			Run the optimization, first with uniform loading
			and then again with tip loads applied.
			What optimized thickness distributions do you see?
			\item Run the optimization with tip loads applied for 
			a range of different mesh sizes (num\_y).
			Plot the computation time vs num\_y.
			\item The script produces an html file, \code{prob1.html},
			that can be useful for studying the problem structure.
			What is the physical interpretation of this problem?
			That is, what are we minimizing and subject to what constraint?
		\ee
		
		\item \textbf{Multidisciplinary analysis.} We now want to couple
		aerodynamics and structures together. We will use 
		\code{prob2.py} for this problem.
		\be
			\item Assemble the aerostructural analysis group following
			\code{aerostruct.html}, and use the nonlinear Gauss--Seidel
			solver to converge the coupled system.
			\item Try NLGS and Newton, and Hybrid NLGS/Newton for 
			mesh sizes (2,3), (4,5), (5,6) and compare run times. 
			Why do we put the nonlinear solver on the `coupled` group, 
			instead of the `root` group?
			\item (Bonus) Try LNGS, Krylov, Krylov-PC-GS, and direct linear solvers 
			with the Newton nonlinear solver. 
			Which ones can successfully converge the linear problem? 
			Which one gives the fastest convergence for the Newton solver?
		\ee
		
		\item \textbf{Multidisciplinary optimization.} We will first compute
		derivatives and the perform optimization of the aerostructural system.
		\be
			\item Compute the derivatives of the multidisciplinary system
			using finite differences by running \code{prob3a.py}.
			\item Now compute the same derivatives using the adjoint method
			and compare the timings for different mesh sizes.
			\item We will now perform aerostructural optimization.
			Edit \code{prob3c.py} by adding the following design variables:
			\begin{itemize}
				\item `twist', lower = -10, upper = 10, scaler = 1000
				\item `alpha', lower = -10, upper = 10, scaler = 1000
				\item `t', lower = 0.003, upper = 0.025, scaler = 1000
			\end{itemize}
			and the follow objective and constraints:
			\begin{itemize}
				\item `fuelburn'
				\item `failure', upper = 0
				\item `eq\_con', equals = 0
			\end{itemize}
		\ee
		
	\ee

\end{document}